%%%%%%%%%%%%%%%%%%%%%%%%%%%%%%%%%%%%%%%%%%%%%%%%%%%%%%%%%%%%%%%
%
%
%%%%%%%%%%%%%%%%%%%%%%%%%%%%%%%%%%%%%%%%%%%%%%%%%%%%%%%%%%%%%%%

\documentclass{article}
\usepackage{graphicx}
\usepackage{fullpage}
\usepackage{amsmath}

\title{Übungsblatt 7}
\author{Tobias Baake (247074), Dylan Ellinger (247316), Nikiforos Tompoulidis (247714)}
\begin{document}
\maketitle

\section{Matrix-Vektor Multiplikation}

\paragraph{a)}

$$ u \cdot \begin{pmatrix} a \\ b \\ c \end{pmatrix} + v \cdot \begin{pmatrix} x \\ y \\ z \end{pmatrix} + w \cdot \begin{pmatrix} r \\ s \\ t \end{pmatrix} = \begin{pmatrix} 3 \\ 4 \\ 5 \end{pmatrix} $$

Die Koeffizienten kann man auch als Vektoren darstellen. Die 3 Vektoren können als 3x3-Matrix dargestellt werden:

$$ \begin{pmatrix} u \\ v \\ w \end{pmatrix} \cdot \begin{pmatrix} a & x & r \\ b & y & s \\ c & z & t \end{pmatrix} = \begin{pmatrix} 3 \\ 4 \\ 5 \end{pmatrix} $$

\paragraph{b)}

\subparagraph{(1)}

$$ \begin{pmatrix} -1 & 3 & -4 \\ 2 & 1 & 2 \\ -3 & 5 & 1 \end{pmatrix} \cdot \begin{pmatrix} 2 \\ 0 \\ -1 \end{pmatrix} = \begin{pmatrix} -1 \cdot 2 + 3 \cdot 0 + (-4) \cdot (-1) \\ 2 \cdot 2 + 1 \cdot 0 + 2 \cdot (-1) \\ -3 \cdot 2 + 5 \cdot 0 + 1 \cdot (-1) \end{pmatrix} = \begin{pmatrix} 2 \\ 2 \\ -7 \end{pmatrix} $$

\subparagraph{(2)}

$$ \begin{pmatrix} 2 & -3 & -4 \\ 1 & 1 & -5 \\ -1 & 0 & 2 \end{pmatrix} \cdot \begin{pmatrix} 3 \\ -2 \\ 0 \end{pmatrix} = \begin{pmatrix} 2 \cdot 3 + (-3) \cdot (-2) + (-4) \cdot 0 \\ 1 \cdot 3 + 1 \cdot (-2) + (-5) \cdot 0 \\ -1 \cdot 3 + 0 \cdot (-2) + 2 \cdot 0 \end{pmatrix} = \begin{pmatrix} 12 \\ 1 \\ -3 \end{pmatrix} $$

\section{Matrix-Matrix Multiplikation}

\paragraph{a)}

Wie bei 7.1 können wir die Koeffizienten umstellen. Hier als Matrix. Dasselbe mit den Ergebnisvektoren. Die Vektoren gruppieren wir zu einer 3x3-Matrix.

$$ \begin{pmatrix} u & v & w \\ l & m & n \end{pmatrix} \cdot \begin{pmatrix} x & a & r \\ y & b & s \\ z & c & t \end{pmatrix} = \begin{pmatrix} 3 & 3 & 5 \\ 1 & 4 & 2 \end{pmatrix} $$

\paragraph{b) (Zeile $\cdot$ Spalte)}

\begin{align*}
    & \begin{pmatrix} -1 & 3 & -4 \\ 2 & 1 & 2 \\ -3 & 5 & 1 \end{pmatrix} \cdot \begin{pmatrix} 2 & -3 & -4 \\ 1 & 1 & -5 \\ -1 & 0 & 2 \end{pmatrix}\\
    & = \begin{pmatrix} 
    -1 \cdot 2 + 3 \cdot 1 + (-4) \cdot (-1) & -1 \cdot (-3) + 3 \cdot 1 + (-4) \cdot 0 & -1 \cdot (-4) + 3 \cdot (-5) + (-4) \cdot 2 \\ 
    2 \cdot 2 + 1 \cdot 1 + 2 \cdot (-1) & 2 \cdot (-3) + 1 \cdot 1 + 2 \cdot 0 & 2 \cdot (-4) + 1 \cdot (-5) + 2 \cdot 2 \\ 
    -3 \cdot 2 + 5 \cdot 1 + 1 \cdot (-1) & -3 \cdot (-3) + 5 \cdot 1 + 1 \cdot 0 & -3 \cdot (-4) + 5 \cdot (-5) + 1 \cdot 2 
    \end{pmatrix}\\
    & = \begin{pmatrix} 
    5 & 6 & -19 \\ 
    3 & -5 & -9 \\ 
    -2 & 14 & -11 
    \end{pmatrix}
\end{align*}

\section{Rotationsmatrix}

\subsection{a) Eigenschaften einer Rotationsmatrix}

\begin{itemize}
    \item Orthogonalität: Eine Rotationsmatrix R ist orthogonal, d.h. $ R^T R = RR^T = I$, wobei $R^T$ die transponierte von $R$ und $I$ die Einheitsmatrix ist
    \item Determinante: Die Determinante einer Rotationsmatrix ist immer gleich 1, d.h., $det(R) = 1$
    \item Erhaltung der Länge: Eine Rotationsmatrix verändert nicht die Länge eines Vektors, d.h., wenn $v$ ein Vektor ist, dann bleibt $||Rv|| = ||v||$
\end{itemize}

\subsection{b) Ergänzen Sie die fehlenden Einträge}

    $$
    \begin{pmatrix}
    \frac{\sqrt{2}}{2} & 0 & ? \\
    0 & ? & ? \\
    ? & ? & ?
    \end{pmatrix} 
    $$
    Angenommen, wir betrachten eine Rotationsmatrix um die z-Achse, so hat sie die Form:
    $$ 
    R_z(\theta) = \begin{pmatrix}
    \cos(\theta) & -\sin(\theta) & 0 \\
    \sin(\theta) & \cos(\theta) & 0 \\
    0 & 0 & 1
    \end{pmatrix} 
    $$
    Angenommen, \(\theta = 45^\circ\) (oder \(\frac{\pi}{4}\) im Bogenmaß), dann ist \(\cos(\theta) = \sin(\theta) = \frac{\sqrt{2}}{2}\). Also sieht die Matrix \( R_z \) so aus:
    $$
    R_z\left(\frac{\pi}{4}\right) = \begin{pmatrix}
    \frac{\sqrt{2}}{2} & -\frac{\sqrt{2}}{2} & 0 \\
    \frac{\sqrt{2}}{2} & \frac{\sqrt{2}}{2} & 0 \\
    0 & 0 & 1
    \end{pmatrix}
    $$
    Nun ergänzen wir die Matrix \( A \) entsprechend:
    $$
    A = \begin{pmatrix}
    \frac{\sqrt{2}}{2} & 0 & 0 \\
    0 & \frac{\sqrt{2}}{2} & -\frac{\sqrt{2}}{2} \\
    0 & \frac{\sqrt{2}}{2} & \frac{\sqrt{2}}{2}
    \end{pmatrix}
    $$
    Da diese Form jedoch nicht die Normale für Rotation um eine Achse ist, kann es besser sein, zu überprüfen, welche Form gegeben wird. In diesem Fall ist die 0 in der letzten Zeile besonders.
    $$
    A = \begin{pmatrix}
    \frac{\sqrt{2}}{2} & 0 & 0 \\
    0 & \frac{\sqrt{2}}{2} & -\frac{\sqrt{2}}{2} \\
    0 & \frac{\sqrt{2}}{2} & \frac{\sqrt{2}}{2}
    \end{pmatrix}
    $$
    In dieser Form bleibt nur die \( A_{22} \) und \( A_{23} \). Somit wäre:
    $$
    A = \begin{pmatrix}
    \frac{\sqrt{2}}{2} & 0 & 0 \\
    0 & \frac{\sqrt{2}}{2} & -\frac{\sqrt{2}}{2} \\
    0 & \frac{\sqrt{2}}{2} & \frac{\sqrt{2}}{2}
    \end{pmatrix}
    $$

\subsection{c)}

Um zu zeigen, dass zwei aufeinanderfolgende 2D-Rotationen additiv sind, müssen wir beweisen, das die Multiplikation von zwei Rotationsmatrizen \(\mathbf{R}(\theta_1)\) und \(\mathbf{R}(\theta_2)\) gleich der Rotationsmatrix \(\mathbf{R}(\theta_1 + \theta_2)\) ist.
\\\\
Eine 2D-Rotationsmatrix \(\mathbf{R}(\theta)\) für einen Winkel \(\theta\) hat die Form:
\\\\
$
\mathbf{R}(\theta) = \begin{pmatrix}
\cos(\theta) & -\sin(\theta) \\
\sin(\theta) & \cos(\theta)
\end{pmatrix}
$
\\\\
Nun betrachten wir die Multiplikation von zwei solchen Matrizen:
\\
$
\mathbf{R}(\theta_1) \mathbf{R}(\theta_2) = \begin{pmatrix}
\cos(\theta_1) & -\sin(\theta_1) \\
\sin(\theta_1) & \cos(\theta_1)
\end{pmatrix}
\begin{pmatrix}
\cos(\theta_2) & -\sin(\theta_2) \\
\sin(\theta_2) & \cos(\theta_2)
\end{pmatrix}
$
\\\\
Durchführen der Multiplikation, unter Bestimmung der Einträge in der Matrix
\\\\
$
\mathbf{R}(\theta_1) \mathbf{R}(\theta_2) = \begin{pmatrix}
\cos(\theta_1)\cos(\theta_2) - \sin(\theta_1)\sin(\theta_2) & -\cos(\theta_1)\sin(\theta_2) - \sin(\theta_1)\cos(\theta_2) \\
\sin(\theta_1)\cos(\theta_2) + \cos(\theta_1)\sin(\theta_2) & -\sin(\theta_1)\sin(\theta_2) + \cos(\theta_1)\cos(\theta_2)
\end{pmatrix}
$
\\\\
Anwendung der Additiontheoreme für Cosinus und Sinus:
\\\\
$
\cos(\theta_1 + \theta_2) = \cos(\theta_1)\cos(\theta_2) - \sin(\theta_1)\sin(\theta_2)
$
\\
$
\sin(\theta_1 + \theta_2) = \sin(\theta_1)\cos(\theta_2) + \cos(\theta_1)\sin(\theta_2)
$
\\
$
-\sin(\theta_1 + \theta_2) = -(\sin(\theta_1)\cos(\theta_2) + \cos(\theta_1)\sin(\theta_2)) = -\sin(\theta_1)\cos(\theta_2) - \cos(\theta_1)\sin(\theta_2)
$
\\
$
-\sin(\theta_1)\sin(\theta_2) + \cos(\theta_1)\cos(\theta_2) = \cos(\theta_1)\cos(\theta_2) - \sin(\theta_1)\sin(\theta_2) = \cos(\theta_1 + \theta_2)
$
\\\\
Damiterhält man folgendes:
\\\\
$
\mathbf{R}(\theta_1) \mathbf{R}(\theta_2) = \begin{pmatrix}
\cos(\theta_1 + \theta_2) & -\sin(\theta_1 + \theta_2) \\
\sin(\theta_1 + \theta_2) & \cos(\theta_1 + \theta_2)
\end{pmatrix} = \mathbf{R}(\theta_1 + \theta_2)
$
\\\\
Das zeigt, dass zwei aufeinanderfolgende 2D-Rotationen additiv sind, also \(\mathbf{R}(\theta_1) \mathbf{R}(\theta_2) = \mathbf{R}(\theta_1 + \theta_2)\).
\\\\
\section{Ratios}

\subsection*{a) Bestimmen Sie das Teilverhältnis \( r = \text{ratio}(p_1, p_2, p_3) \) für die folgenden Punkte:}

1. Für die Anordnung \( p_1, p_2, p_3 \):

\[ r = \frac{p_1 - p_2}{p_3 - p_2} = \frac{0 - 1}{2 - 1} = \frac{-1}{1} = -1 \]

2. Für die Anordnung \( p_2, p_3, p_1 \):

\[ r = \frac{p_2 - p_3}{p_1 - p_3} = \frac{1 - 2}{0 - 2} = \frac{-1}{-2} = \frac{1}{2} \]

3. Für die Anordnung \( p_3, p_1, p_2 \):

\[ r = \frac{p_3 - p_1}{p_2 - p_1} = \frac{2 - 0}{1 - 0} = \frac{2}{1} = 2 \]

4. Für die Anordnung \( p_3, p_2, p_1 \):

\[ r = \frac{p_3 - p_2}{p_1 - p_2} = \frac{2 - 1}{0 - 1} = \frac{1}{-1} = -1 \]

\subsection*{b) Bestimmen Sie das Doppelverhältnis \( cr = \text{crossratio}(p_1, p_2, p_3, p_4) \) für die folgenden Punkte:}

1. Für die Anordnung \( p_1, p_2, p_3, p_4 \):

\[ cr = \frac{(p_1 - p_3)(p_2 - p_4)}{(p_1 - p_4)(p_2 - p_3)} = \frac{(0 - 2)(1 - 3)}{(0 - 3)(1 - 2)} = \frac{(-2)(-2)}{(-3)(-1)} = \frac{4}{3} \]

2. Für die Anordnung \( p_1, p_3, p_4, p_2 \):

\[ cr = \frac{(p_1 - p_4)(p_3 - p_2)}{(p_1 - p_2)(p_3 - p_4)} = \frac{(0 - 3)(2 - 1)}{(0 - 1)(2 - 3)} = \frac{(-3)(1)}{(-1)(-1)} = \frac{-3}{1} = -3 \]

3. Für die Anordnung \( p_4, p_1, p_2, p_3 \):

\[ cr = \frac{(p_4 - p_2)(p_1 - p_3)}{(p_4 - p_3)(p_1 - p_2)} = \frac{(3 - 1)(0 - 2)}{(3 - 2)(0 - 1)} = \frac{(2)(-2)}{(1)(-1)} = \frac{-4}{-1} = 4 \]

4. Für die Anordnung \( p_1, p_4, p_3, p_2 \):

\[ cr = \frac{(p_1 - p_3)(p_4 - p_2)}{(p_1 - p_2)(p_4 - p_3)} = \frac{(0 - 2)(3 - 1)}{(0 - 1)(3 - 2)} = \frac{(-2)(2)}{(-1)(1)} = \frac{-4}{-1} = 4 \]

\end{document}
    
\end{document}
