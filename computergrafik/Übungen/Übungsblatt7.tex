%%%%%%%%%%%%%%%%%%%%%%%%%%%%%%%%%%%%%%%%%%%%%%%%%%%%%%%%%%%%%%%
%
%
%%%%%%%%%%%%%%%%%%%%%%%%%%%%%%%%%%%%%%%%%%%%%%%%%%%%%%%%%%%%%%%

\documentclass{article}
\usepackage{graphicx}
\usepackage{fullpage}
\usepackage{amsmath}

\title{Übungsblatt 7}
\author{Tobias Baake (247074), Dylan Ellinger (247316), Nikiforos Tompoulidis (247714)}
\begin{document}
\maketitle

\section{Matrix-Vektor Multiplikation}

\paragraph{a)}

$$ u \cdot \pmatrix{a \cr b \cr c} + v \cdot  \pmatrix{x \cr y \cr z} + w \cdot \pmatrix{r \cr s \cr t} = \pmatrix{3 \cr 4 \cr 5}$$

Die Koeffizienten kann man auch als Vektoren darstellen
Die 3 Vektoren können als 3x3 Matrix dargestellt werden

$$\pmatrix{u \cr v \cr w} \cdot \pmatrix{a & x & r \cr b & y & s \cr c & z & t} = \pmatrix{3 \cr 4 \cr 5}$$

\paragraph{b)}

\subparagraph{(1)}

$$\pmatrix{-1 & 3 & -4 \cr 2 & 1 & 2 \cr -3 & 5 & 1} \cdot \pmatrix{2 \cr 0 \cr -1} = \pmatrix{-1 \cdot 2 & + & 3 \cdot 0 & + & (-4) \cdot (-1) \cr 2 \cdot 2 & + & 1 \cdot 0 & + & 2 \cdot (-1) \cr -3 \cdot 2 & + & 5 \cdot 0 & + & 1 \cdot (-1)} = \pmatrix{2 \cr 2 \cr -7}$$

\subparagraph{(2)}

$$\pmatrix{2 & -3 & -4 \cr 1 & 1 & -5 \cr -1 & 0 & 2} \cdot \pmatrix{3 \cr -2 \cr 0} = \pmatrix{2 \cdot 3 & + & (-3) \cdot (-2) & + & (-4) \cdot 0 \cr 1 \cdot 3 & + & 1 \cdot (-2) & + & (-5) \cdot 0 \cr -1 \cdot 3 & + & 0 \cdot (-2) & + & 2 \cdot 0} = \pmatrix{12 \cr 1 \cr -3}$$

\section{Matrix-Matrix Multiplikation}

\paragraph{a)}

Wie bei 7.1 könne wir die Koeffizienten umstellen. Hier als Matirx. Das selbe mit den Ergebnisvektoren. Die Vektoren gruppieren wir zu einer 3x3 Matrix.

$$\pmatrix{u & v & w \cr l & m & n} \cdot \pmatrix{x & a & r \cr y & b & s \cr z & c & t} = \pmatrix{3 & 3 & 5 \cr 1 & 4 & 2}$$

\paragraph{b)}

$$

\end{document}

