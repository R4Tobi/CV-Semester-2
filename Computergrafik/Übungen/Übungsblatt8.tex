
\documentclass{article}
\usepackage{graphicx}
\usepackage{fullpage}
\usepackage{amsmath}

\title{Übungsblatt 8}
\author{Tobias Baake (247074), Dylan Ellinger (247316), Nikiforos Tompoulidis (247714)}
\begin{document}
\maketitle

\section{Transformationsmatrizen}
\subsection*{a) $y= 2x-5$}

Spiegelung der Gerade y beinhaltet mehrere Schritte
\begin{itemize}
    \item Translation zum Ursprung
    \item Rotation zur x-Achse
    \item Spiegelung
    \item Rückrotation
    \item Rücktranslation $\rightarrow T(x_1, y_1) \cdot R(a) \cdot S_p \cdot R^{-1}(\alpha) \cdot T(-x_1, -y_1)$
\end{itemize}

Transformationsmatrix:
$$
\begin{pmatrix}
    1 & 0 & 0 \\
    0 & 1 & 5 \\
    0 & 0 & 1 \\
\end{pmatrix}
\cdot
\begin{pmatrix}
    \cos(\alpha) & -\sin(\alpha) & 0 \\
    \sin(\alpha) & \cos(\alpha) & 0 \\
    0 & 0 & 1 \\
\end{pmatrix}
\cdot
\begin{pmatrix}
    1 & 0 & 0 \\
    0 & -1 & 0 \\
    0 & 0 & 1 \\
\end{pmatrix}
\cdot
\begin{pmatrix}
    \cos(\alpha) & \sin(\alpha) & 0 \\
    -\sin(\alpha) & \cos(\alpha) & 0 \\
    0 & 0 & 1 \\
\end{pmatrix}
\cdot
\begin{pmatrix}
    1 & 0 & 0 \\
    0 & 1 & -5 \\
    0 & 0 & 1 \\
\end{pmatrix}
$$

\subsection*{b) Affine Transformation 2d:}
$$
\begin{pmatrix}
    x' \\
    y' \\
    w \\
\end{pmatrix}
=
\begin{pmatrix}
    a & b & c \\
    d & e & f \\
    0 & 0 & 1 \\
\end{pmatrix}
\cdot
\begin{pmatrix}
    x \\
    y \\
    1 \\
\end{pmatrix}
$$
Zuerst muss man die Original-Bildpunkt-Paare homogen darstellen:
$$
x_1 =
\begin{pmatrix}
    0 \\
    0 \\
    1 \\
\end{pmatrix}
\rightarrow x'_1 =
\begin{pmatrix}
    1 \\
    1 \\
    1 \\
\end{pmatrix}, 
x_2 =
\begin{pmatrix}
    1 \\
    0 \\
    1 \\
\end{pmatrix}
\rightarrow x'_2 =
\begin{pmatrix}
    3 \\
    2 \\
    1 \\
\end{pmatrix},
x_3 =
\begin{pmatrix}
    1 \\
    2 \\
    1 \\
\end{pmatrix}
\rightarrow x'_3 =
\begin{pmatrix}
    9 \\
    -6 \\
    1 \\
\end{pmatrix}
$$
Paare als Matrizen aufchreiben:
$$
\begin{pmatrix}
    1 & 3 & 9 \\
    1 & 2 & -6 \\
    1 & 1 & 1 \\
\end{pmatrix}
=
\begin{pmatrix}
    a & b & c \\
    d & e & f \\
    0 & 0 & 1 \\
\end{pmatrix}
\cdot
\begin{pmatrix}
    0 & 1 & 1 \\
    0 & 0 & 2 \\
    1 & 1 & 1 \\
\end{pmatrix}
$$
Nun kann man immer ewinzeln die Variablen durch das multißplizieren der Zeilen der einen Matrix mit den Spalten der anderen bestimmen:
\subsubsection*{Element 1.1}

\begin{align*}
    a \cdot 0 + b \cdot 0 + c \cdot 1 = 1 \\
    c = 1
\end{align*}

\subsubsection*{Element 1.2}

\begin{align*}
    a \cdot 1 + b \cdot 0 + c \cdot 1 = 3 \\
    a + c = 3 \\
    a + 1 = 3 \\
    a = 2
\end{align*}

\subsubsection*{Element 1.3}

\begin{align*}
    a \cdot 1 + b \cdot 2 + c \cdot 1 = 9 \\
    2 + 2b + 1 = 9 \\
    2b = 6 \\
    b = 3
\end{align*}

\subsubsection*{Element 2.1}

\begin{align*}
    (d \cdot 0 + e \cdot 0 + f \cdot 1) = 1 \\
    f = 1
\end{align*}

\subsubsection*{Element 2.2}

\begin{align*}
    (d \cdot 1 + e \cdot 0 + f \cdot 1) = 2 \\
    d + f = 2 \\
    d + 1 = 2 \\
    d = 1
\end{align*}

\subsubsection*{Element 2.3}

\begin{align*}
    (d \cdot 1 + e \cdot 2 + f \cdot 1) = -6 \\
    1 + 2e + 1 = -6 \\
    2e = -8 \\
    e = -4
\end{align*}
Jetzt sind alle Werte bestimmt:
$$
a = 2, b = 3, c = 1, d = 1, e = -4, f = 1
$$
Einsetzen in die Transformationsmatrix:
$$
\underline{\underline{
B =
\begin{pmatrix}
    2 & 3 & 1 \\
    1 & -4 & 1 \\
    0 & 0 & 1 \\
\end{pmatrix}
}}
$$
\subsection*{c)}
Wir können die affinen Transformation in homogenen Koordinaten nutzen. \\
C muss in der Form $C = \begin{pmatrix} a & b & c \\ d & e & f \\ 0 & 0 & 1 \end{pmatrix}$ gegeben sein. \\
Die Transformationen sind bereits in der Aufgabenstellung gegeben.\\
Es müssen also nur noch die Werte eingesetzt werden.\\
$$
\underline{\underline{C = \begin{pmatrix} \pi & \cos ^2(\alpha) & -3 \\ \tan(\alpha) & - \sin ^2(\alpha) & \frac{\pi}{2}-2 \\ 0 & 0 & 1 \end{pmatrix}}}
$$

\subsection*{d)}
Gegebene Werte als homogene Koordinaten umschreiben
$$
\begin{pmatrix}
    x \\
    y \\
    1 \\
\end{pmatrix}
\rightarrow D
\begin{pmatrix}
    x \\
    y \\
    1 \\
\end{pmatrix}
$$
$$
\begin{pmatrix}
    x \\
    y \\
    1 \\
\end{pmatrix}
\rightarrow d
\begin{pmatrix}
    3 \\
    \frac{\pi \cdot x}{y -1} + \pi \\
    1 \\
\end{pmatrix}
$$
Erste Komponente der Zielabbildung ist 3. Sie ist konstant.\\
\\
Die Zweite KOmponente ist etwas komplizierter mit $\frac{\pi \cdot x}{y -1} + \pi$, aber durch probieren kann man auf die Lösung schließen.\\
Hier dasselbe Prinzip wie bei den vorherigen Aufgaben:
$$
D \cdot 
\begin{pmatrix}
    x \\
    y \\
    1 \\
\end{pmatrix}
= 
\begin{pmatrix}
    a & b & c \\
    d & e & f \\
    0 & 0 & 1 \\
\end{pmatrix}
\cdot
\begin{pmatrix}
    x \\
    y \\
    1 \\
\end{pmatrix}
=
\begin{pmatrix}
    3 \\
    \frac{\pi \cdot x}{y -1} + \pi \\
    1 \\
\end{pmatrix}
$$
Selbes Prinzip wie immer: Zeile mal SPalte für das Ergebnis.
Damit wir 3 bekommen setzen wir $a=0, b=0, c=3$ damit:
$$
(0 \cdot x + 0 \cdot y + 3 \cdot 1) = 3
$$
Wie bereits erwähnt ist die zweite Zeile komplizierter:\\
d und f sind offensichtlich, denn wenn $d = \pi$ und $f = \pi$ führt zu $x \cdot \pi = \pi x$ und $\pi \cdot 1 = \pi$\\
Für $e$ stellen wir den Ausdruck um.\\
Man zieht $\pi$ als gemeinsamen Faktor aus beiden Termen heraus.
$$
\frac{\pi \cdot x}{y -1} + \pi = \pi \cdot \left( \frac{x}{y -1} + 1 \right)
$$
so ist es leicht $e$ als $-\pi$ zu bestimmen.\\
Wir erhalten als Lösung folgende 3x3 Matrix D:
$$
\underline{\underline{
    D =
    \begin{pmatrix}
        0 & 0 & 3 \\
        \pi & -\pi & \pi \\
        0 & 0 & 1 \\
    \end{pmatrix}
}}
$$
\section{Komposition von Transformationen}
Die Eckpunkte des blauen Rechtecks sind:
\begin{align*}
a &= (1, 2) \\
b &= (2, 2) \\
c &= (1, 3) \\
d &= (2, 3)
\end{align*}

Die Eckpunkte des orangefarbenen Rechtecks sind:
\begin{align*}
a' &= (11, 4.5) \\
b' &= (7, 8.5) \\
c' &= (6, 7.5) \\
d' &= (10, 3.5)
\end{align*}

\subsubsection*{Bestimmung der Transformation}

\paragraph{1. Berechnung der Skalierung:}
\begin{itemize}
    \item Die Seitenlänge des blauen Rechtecks beträgt 1.
    \item Die Seitenlänge des orangefarbenen Rechtecks beträgt \( \sqrt{4.5^2 + 4.5^2} = 4.5\sqrt{2} \).
    \item Daher beträgt die Skalierung \( s = 4.5\sqrt{2} \).
\end{itemize}

\paragraph{2. Berechnung der Drehung:}
\begin{itemize}
    \item Das Rechteck wurde um \( 45^\circ + 90^\circ = 135^\circ \) gegen den Uhrzeigersinn gedreht.
    \item Die Drehmatrix für eine Drehung um \( \theta \) ist:
    \[
    R = \begin{pmatrix}
    \cos\theta & -\sin\theta \\
    \sin\theta & \cos\theta
    \end{pmatrix}
    \]
    \item Für \( \theta = 135^\circ \):
    \[
    R = \begin{pmatrix}
    -\frac{1}{\sqrt{2}} & -\frac{1}{\sqrt{2}} \\
    \frac{1}{\sqrt{2}} & -\frac{1}{\sqrt{2}}
    \end{pmatrix}
    \]
\end{itemize}

\paragraph{3. Berechnung der Gesamttransformation:}
\begin{itemize}
    \item Die Skalierungs- und Drehmatrix wird kombiniert:
    \[
    T = sR = 4.5\sqrt{2} \begin{pmatrix}
    -\frac{1}{\sqrt{2}} & -\frac{1}{\sqrt{2}} \\
    \frac{1}{\sqrt{2}} & -\frac{1}{\sqrt{2}}
    \end{pmatrix} = \begin{pmatrix}
    -4.5 & -4.5 \\
    4.5 & -4.5
    \end{pmatrix}
    \]
\end{itemize}

\paragraph{4. Homogene Matrix:}
Um die homogene Transformation zu erhalten, fügen wir die Translation hinzu:
\[
T = \begin{pmatrix}
-4.5 & -4.5 & 0 \\
4.5 & -4.5 & 0 \\
11 & 4.5 & 1
\end{pmatrix}
\]
Diese Matrix kombiniert die Skalierung, Drehung und Translation in einer einzigen Transformation.

\section{Affine und Projektive Abbildungen}
\subsection*{a)}
projektiv und affin, da parralellität und kollineraität erhalten bleibt. Eine Drehung um 45° ist eine projektive Transformation, da sie durch eine Kombination aus einer Rotation und einer Translation erreicht werden kann.
\subsection*{b)}
Diese Transformation ist projektiv, aber nicht affin, da die Parallellität nicht erhalten bleibt. Die Punkte bleiben zwar auf einer Linie, dennoch werden sie verzerrt.
\subsection*{c)}
Diese Transformation ist weder affin noch projektiv, da die Kollinearität und Parallellität nicht erhalten bleiben. Die Punkte bleiben werder aufeiner Linie, noch behalten sie ihre Abstände zueinander.
\subsection*{d)}
Diese Transformation ist affin, da sie durch eine Kombination aus einer Translation, Skalierung (relative Skalierung) und Rotation erreicht werden kann. Sie ist auch projektiv, da sie Kollinearität erhält.
\subsection*{e)}
Diese Transformation vertauscht die Punkte und verzerrt ihre Positionen. Diese Transformation ist daher weder affin noch projektiv, da die Parallellität und Kollinearität nicht erhalten bleiben.
\subsection*{f)}
Diese Transformation ist projektiv, da sie Kollinearität erhält, jedoch nicht die Paraalelität.
\end{document}